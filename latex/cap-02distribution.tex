%% ------------------------------------------------------------------------- %%
\chapter{The endpoint-inflated beta-binomial distribution}
\label{cap:distribution}

In this chapter we show the steps to construct the endpoint-inflated beta-binomial distribution, examine its properties and define the parametrization that will be used through the rest of this paper.

We begin by introducing the simpler and familiar distributions that will be used as our building blocks.

%% ------------------------------------------------------------------------- %%
\section{Binomial distribution}
\label{sec:bino-dist}

When $X$ indicates the total number of successes in a series of $n \in \mathbb{N}^+$ independent dichotomous trials, each with probability $p \in [0,1]$, we say that it has a binomial distribution. The point mass function of a binomial random variable is given by

\begin{equation}
\label{binomial-pmf}
\begin{split}
f_{X}(x \mid p; n)
&= \binom{n}{x}p^x(1-p)^{n-x}\\
\end{split}
\end{equation}

with central moments

\begin{equation}
\label{binomial-moments}
\begin{split}
\e{X} = np, \quad \var{X} = np(1-p).
\end{split}
\end{equation}

The binomial upper bound $n$ will be assumed to be known throughout this paper and excluded from discussions of the distribution's parameters, though the opposite can hold in more general treatments.

%% ------------------------------------------------------------------------- %%
\section{Beta distribution}
\label{sec:beta-dist}

A random variable $Y \in (0,1)$ follows a beta distribution with parameters $\alpha, \beta > 0$ if its probability density function is given by

\begin{equation}
\begin{split}
f_{Y}(y \mid \alpha, \beta)
&= \frac{\Gamma(\alpha+\beta)}{\Gamma(\alpha)\Gamma(\beta)}y^{\alpha-1}(1-y)^{\beta-1},
\end{split}
\end{equation}

which central moments $\e{Y} = \alpha / (\alpha+ \beta)$ and $\var{Y} = \alpha \beta/[(\alpha+\beta)^2(\alpha+\beta+1)]$.

For brevity, expressions that take the form of the normalizing factor in the above pdf will, from this point on, be denoted through the beta function:

\begin{equation}
\label{beta-function}
\begin{split}
B(\alpha, \beta) = \frac{\Gamma(\alpha)\Gamma(\beta)}{\Gamma(\alpha+\beta)}
\end{split}
\end{equation}

In the regression context that will be developed later, it is more convenient to parametrize the beta distribution in terms of its mean $\mu \in (0,1)$ and a precision parameter $\phi > 0$. The resulting equivalence is  $\alpha = \mu\phi$ and $\beta = (1-\mu)\phi$ and the central moments are reexpressed as

\begin{equation}
\begin{split}
\e{Y} = \mu, \quad \var{Y} = \frac{\mu(1-\mu)}{\phi+1}.
\end{split}
\end{equation}

%% ------------------------------------------------------------------------- %%
\section{Beta-binomial distribution}
\label{sec:bbin-dist}

Beta random variables have support over $(0,1)$, which includes all values that are allowed for the $p$ parameter in a non-degenerate binomial random variable. If observations on a binomial random variable $X$ are thought to come from a randomly drawn, beta-distributed $p=P$, then the marginal distribution of $X$ is beta-binomial. Notationally, the relationship is

\begin{equation}
\begin{split}
X \mid P &\sim \text{Binomial}(P; n)\\
P &\sim \text{Beta}(\mu, \phi)\\
\Rightarrow X &\sim \text{Beta-binomial}(\mu, \phi; n).
\end{split}
\end{equation}

With $B(\alpha,\beta)$ as defined in \ref{beta-function}, the beta-binomial pmf is

\begin{equation}
\begin{split}
f_X(x \mid \mu, \phi; n) = \binom{n}{x}\frac{B(x+\mu\phi, n - x + (1-\mu)\phi)}{B(\mu\phi, (1-\mu)\phi)},
\end{split}
\label{betabinomial-pmf}
\end{equation}

with central moments

\begin{equation}
\begin{split}
\e{X} = n\mu, \quad \var{X} = n\mu(1-\mu)\frac{\phi+n}{\phi+1}.
\end{split}
\end{equation}

Comparing the above expressions with those for the binomial \ref{binomial-moments}, it can be seen that the meas take the same form, both being governed by a single centrality parameter on the unit iterval. For identical values of $\mu$ and $p$, it can be seen that the beta-binomial variance will exceed that of the binomial by a factor of $(\phi+n)/(\phi+1)$. REFERENCIA A FIGURA CON BINOM VS BETA-BINOM.

%% ------------------------------------------------------------------------- %%
\section{Endpoint-inflated beta-binomial distribution (EIBB)}
\label{sec:eibb-dist}

A random variable $X$ that follows an EIBB distribution takes on the values

sim sim sim

Let $I_c(x)$ be an indicator function for point $c$ and have $Z \sim \mathrm{BB}(\alpha, \beta; n)$, denoting a discrete random variable with beta-binomial (BB) distribution. Then a random variable $Y$ is said to have an endpoint-inflated BB distribution if its pmf is

\begin{equation}
\begin{split}
f_{Y}(y \mid \alpha , \beta , p^{_0}, p^{_1}; n)
&= p^{_0}I_0(y) + p^{_1}I_n(y) + p^{_2} f_{Z}(y)\\
&=	\begin{cases}
p^{_0} + p^{_2} {B(\alpha, n+\beta) \over B(\alpha,\beta)}, & \mathrm{if}\; y=0\\
p^{_2} {n \choose y} {B(y+\alpha, n-y+\beta) \over B(\alpha,\beta)}, & \mathrm{if}\; y=1,...,n-1\\
p^{_1} + p^{_2} {B(n+\alpha, \beta) \over B(\alpha,\beta)}, & \mathrm{if}\; y=n\\
0, & \mathrm{otherwise}
	\end{cases}
\end{split}
\label{densidad}
\end{equation}
	
\noindent where the shape parameters $\alpha > 0$ and $\beta > 0$ are unknown, as are $p^{_0} \in (0, 1)$ and $p^{_1} \in (0, 1)$, which represent the proportions of left an right endpoints that occur in excess of that which would be expected if the generating distribution was beta-binomial alone. Clearly, $p^{_2} = 1 -  (p^{_0}+p^{_1})$.  The right endpoint, $n$, is assumed to be known.\\

\section{Mean and precision reparametrization}
\label{sec:reparam}

We parametrize the BB component of the distribution in terms of its mean, $\mu \in (0,1)$, and precision, $\phi > 0$. The shape parameters can then be expressed as $\alpha = \mu\phi$ and $\beta = (1-\mu)\phi$. This results in a simple expression for the mean, $E(Y) = n\mu$, which allows us to model it in terms of a single, easily interpretable parameter. Note that the variance, $V(Y) = n\mu(1-\mu)(\phi+n)/(\phi+1)$, approaches a binomial variance for large $\phi$ and is bounded for a given value of $n$.\\

\begin{figure}
  \includegraphics[width=\linewidth]{eibb2.png}
  \caption{The EIBB distribution for three combinations of mean and precision parameters with fixed mixture proportions $p^{_0} = p^{_1} = 0.25$}
  \label{fig:eibb}
\end{figure}

\section{Identifiability}
\label{sec:ident}

In \cite{Dupuy2017}, conditions for identifiability of the closely related endpoint-inflated binomial likelihood are given. Briefly, all three componentes of the mixture must have nonzero values and all observed $Y_i$ should have a corresponding $n_i \in \{3,...,M\}$ for some finite integer $M$.

Providing a similar result for the beta-binomial case is complicated by the presence of parameters inside gamma functions. However, in the following sections it will be shown via simulations that even when the likelihood alone is not strictly identifiable, a combination of suitable priors and data allows for successful posterior inference.