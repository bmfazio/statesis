%% ------------------------------------------------------------------------- %%
\chapter{The endpoint-inflated beta-binomial distribution}
\label{cap:distribution}

In this chapter we show the steps to construct the endpoint-inflated beta-binomial distribution, examine its properties and define the parametrization that will be used through the rest of this paper.

We begin by introducing the simpler and familiar distributions that will be used as our building blocks.

%% ------------------------------------------------------------------------- %%
\section{Binomial distribution}
\label{sec:bino-dist}

When $X$ indicates the total number of successes in a series of $n \in \mathbb{N}^+$ independent dichotomous trials, each with probability $p \in [0,1]$, we say that it has a binomial distribution. The point mass function of a binomial random variable is

\begin{equation}
\label{binomial-pmf}
\begin{split}
f_{X}(x \mid p; n)
&= \binom{n}{x}p^x(1-p)^{n-x},\quad x=0,..,n\\
\end{split}
\end{equation}

with mean and variance given by

\begin{equation}
\label{binomial-meanvar}
\begin{split}
\e{X} = np, \quad \var{X} = np(1-p).
\end{split}
\end{equation}

The binomial upper bound $n$ will be assumed to be known throughout this paper and excluded from discussions of the distribution's parameters, though the opposite can hold in more general treatments.

%% ------------------------------------------------------------------------- %%
\section{Beta distribution}
\label{sec:beta-dist}

A random variable $Y \in (0,1)$ follows a beta distribution with parameters $\alpha, \beta > 0$ if its probability density function is given by

\begin{equation}
\begin{split}
f_{Y}(y \mid \alpha, \beta)
&= \frac{\Gamma(\alpha+\beta)}{\Gamma(\alpha)\Gamma(\beta)}y^{\alpha-1}(1-y)^{\beta-1},\quad y \in (0,1)
\end{split}
\end{equation}

with mean $\e{Y} = \alpha / (\alpha+ \beta)$ and variance $\var{Y} = \alpha \beta/[(\alpha+\beta)^2(\alpha+\beta+1)]$.

For brevity, the normalizing factor in the above pdf will from this point on be expressed through the equivalent beta function:

\begin{equation}
\label{beta-function}
\begin{split}
B(\alpha, \beta) = \frac{\Gamma(\alpha)\Gamma(\beta)}{\Gamma(\alpha+\beta)}
\end{split}
\end{equation}

In the regression context that will be developed later, it is more convenient to parametrize the beta distribution in ters of its mean $\mu \in (0,1)$ and a precision parameter $\phi > 0$. The resulting equivalence is  $\alpha = \mu\phi$ and $\beta = (1-\mu)\phi$ and the central moments are reexpressed as

\cite{ferrari2004beta}

\begin{equation}
\begin{split}
\e{Y} = \mu, \quad \var{Y} = \frac{\mu(1-\mu)}{\phi+1}.
\end{split}
\end{equation}

%% ------------------------------------------------------------------------- %%
\section{Beta-binomial distribution}
\label{sec:bbin-dist}

Beta random variables have support over $(0,1)$, which includes all values that are allowed for the $p$ parameter in a non-degenerate binomial random variable. If repeated measurements on a binomial random variable $X$ are thought to reflect a randomly drawn, beta-distributed $p=Y$, then the marginal distribution of $X$ is beta-binomial. Notationally, the relationship is

\begin{equation}
\begin{split}
X \mid Y &\sim \text{Binomial}(p=Y; n)\\
Y &\sim \text{Beta}(\mu, \phi)\\
\Rightarrow X &\sim \text{Beta-binomial}(\mu, \phi; n).
\end{split}
\end{equation}

With $B(\alpha,\beta)$ as defined in \ref{beta-function}, the beta-binomial pmf is

\begin{equation}
\begin{split}
f_X(x \mid \mu, \phi; n) = \binom{n}{x}\frac{B(x+\mu\phi, n - x + (1-\mu)\phi)}{B(\mu\phi, (1-\mu)\phi)},
\end{split}
\label{betabinomial-pmf}
\end{equation}

with central moments

\begin{equation}
\begin{split}
\e{X} = n\mu, \quad \var{X} = n\mu(1-\mu)\frac{\phi+n}{\phi+1}.
\end{split}
\end{equation}

Comparing the above expressions with those for the binomial \ref{binomial-meanvar}, it can be seen that the meas take the same form, both being governed by a single centrality parameter on the unit iterval. For identical values of $\mu$ and $p$, it can be seen that the beta-binomial variance will exceed that of the binomial by a factor of $(\phi+n)/(\phi+1)$. REFERENCIA A FIGURA CON BINOM VS BETA-BINOM.

%% ------------------------------------------------------------------------- %%
\section{Endpoint-inflated beta-binomial distribution (EIBB)}
\label{sec:eibb-dist}

A random variable $X$ that follows an EIBB distribution behaves as

\begin{equation}
\begin{split}
X \mid Z \sim
\begin{cases}
\text{Degenerate}(0) \qquad &\text{ if } Z=0,\\
\text{Beta-binomial}(\mu, \phi; n) \qquad &\text{ if } Z=1,\\
\text{Degenerate}(n) \qquad &\text{ if } Z=2
\end{cases}\\
Z \sim \text{Categorical}(p^{_0}, p^{_1}, p^{_2}),
\end{split}
\label{eibb-distribution}
\end{equation}

where $Z$ is a discrete latent variable. This represents a mixture involving the beta-binomial distribution and two degenerate distributions at the beta-binomial endpoints $0$ and $n$.

In order to write the pmf, we define $I_c(x)$ to be the indicator function for point $c$ and introduce $Y \sim \text{Beta-binomial}(\mu, \phi; n)$. Then the EIBB random variable $X$ has a distribution given by

\begin{equation}
\begin{split}
f_{X}(x \mid \mu, \phi, p^{_0}, p^{_1}, p^{_2}; n)
&= p^{_0}I_0(x) + p^{_1} f_{Y}(x) + p^{_2}I_n(x)\\
&=	\begin{cases}
p^{_0} + p^{_1} {B(\mu\phi, n+(1-\mu)\phi) \over B(\mu\phi,(1-\mu)\phi)} & \text{if}\; x=0\\
p^{_1} {n \choose x} {B(x+\mu\phi, n-x+(1-\mu)\phi) \over B(\mu\phi,(1-\mu)\phi)} & \text{if}\; x=1,...,n-1\\
p^{_2} + p^{_1} {B(n+\mu\phi, (1-\mu)\phi) \over B(\mu\phi,(1-\mu)\phi)} & \text{if}\; x=n\\
0 & \text{otherwise.}
	\end{cases}
\end{split}
\label{densidad}
\end{equation}

We will not require expressions for this distribution's mean and variance, but present them here for completeness:

\begin{equation}
\begin{split}
\e{X} = n(p^{_1}\mu + p^{_2}), \quad \var{X} = n^2\left[ p^{_1}\left( \mu + \frac{ \mu(1-\mu)}{n}\frac{(\phi + n)}{(\phi+1)} \right) + p^{_2} - (p^{_1}\mu + p^{_2})^2\right].
\end{split}
\end{equation}

\begin{figure}
  \includegraphics[width=\linewidth]{eibb2.png}
  \caption{The EIBB distribution for three combinations of mean and precision parameters with fixed mixture proportions $p^{_0} = p^{_1} = 0.25$}
  \label{fig:eibb}
\end{figure}
