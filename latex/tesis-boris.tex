
\documentclass[11pt,oneside,a4paper]{book}

% ---------------------------------------------------------------------------- %

\usepackage[english]{babel}
\usepackage[latin1]{inputenc}
\usepackage[pdftex]{graphicx}           % para insertar figuras en formato pdf/png/jpg
\usepackage{color}
\usepackage{pifont}
\usepackage{amsfonts}
\usepackage{amssymb} 
\usepackage{amsmath}
\usepackage{setspace}                   % espaciamento flexible
\usepackage[small,compact]{titlesec} 	% encabezamiento de los titulos: menores y compactos
\usepackage{indentfirst} 				% indentacion del primer parrafo
\usepackage[round]{natbib} % libreria para la bibliografia
\usepackage{subfigure} 					% uso de varias figuras en una sola
\usepackage[nottoc]{tocbibind} 			% para que la bibliografia aparezca en el indice
\usepackage{setspace}
\usepackage{longtable}
\usepackage{lscape}
\usepackage{caption}
\usepackage[colorlinks=true,urlcolor=red,citecolor=green,linkcolor=blue]{hyperref}
\usepackage[a4paper,top=2.54cm, bottom=2.54cm, left=3cm, right=2.54cm]{geometry} %margenes

% ---------------------------------------------------------------------------- %
% Algunos comandos
\graphicspath{{./img/}} 			% direccion donde se encontraran las figuras
\makeindex  							% para crear el indice
\raggedbottom   					% para no tener espacios extras en el texto
\listfiles  							% genera una lista de los archivos utilizados durante la compilacion
\normalsize

% Para tener un tamaoo de fuente menor (para figuras e cuadros)
\newcommand{\captionfonts}{\small}
\makeatletter  % Allow the use of @ in command names
\long\def\@makecaption#1#2{%
  \vskip\abovecaptionskip
  \sbox\@tempboxa{{\captionfonts #1: #2}}%
  \ifdim \wd\@tempboxa >\hsize
    {\captionfonts #1: #2\par}
  \else
    \hbox to\hsize{\hfil\box\@tempboxa\hfil}%
  \fi
  \vskip\belowcaptionskip}
\makeatother   % Cancel the effect of \makeatletter

% para mejorar la presentacioon de las figuras en el texto
\renewcommand{\topfraction}{0.85}
\renewcommand{\textfraction}{0.1}
\renewcommand{\floatpagefraction}{0.75}

% comodidades
\newcommand{\e}[1]{{\mathbb E}\left[ #1 \right]}
\newcommand{\var}[1]{{\mathbb V}\left[ #1 \right]}

% ---------------------------------------------------------------------------- %
% Cuerpo del texto
\begin{document}
\frontmatter \onehalfspacing  % tipo de interlineado

% ---------------------------------------------------------------------------- %
% Carotula
\thispagestyle{empty}
\begin{center}
\vspace*{1cm}
\textbf{\Large{PONTIFICIA UNIVERSIDAD CAT\'OLICA DEL PER\'U }}\\
\vspace*{1.2cm}

\textbf{\Large{ESCUELA DE GRADUADOS}}\\
\vspace*{0.5cm}
\begin{center}
\includegraphics[scale=.25]{logoPUCP}
\end{center}
\vspace{0.5cm}

\textbf{\Large{``Endpoint-inflated beta-binomial regression\\for correlated count data''}}\\
\vspace{1.2cm}
\textbf{\large{TESIS PARA OPTAR POR EL GRADO DE MAGISTER EN\\
  ESTAD\'ISTICA}}\\
  
\vspace*{1.2cm}
\textbf{\large{Presentado por:}}\\
\vspace*{0.3cm}
\textbf{\large{Boris Manuel Fazio Luna}}\\
\vspace*{1.2cm}
\textbf{\large{Asesor: Victor Giancarlo Sal Y Rosas Celi}}\\

\vspace*{1.2cm}

\textbf{\large{
	Miembros del jurado:\\
	Dr. Nombre completo jurado 1 \\
	Dr. Nombre completo jurado 2 \\
	Dr. Nombre completo jurado 3  
	}} 
	   
\vspace*{1.2cm}
    
\normalsize{Lima, Julio 2018}
\end{center}

%-------------------------------------------
% Dedicatoria
%\chapter*{Dedicatoria}
%Dedicatoria...

%-------------------------------------------
% Agradecimentos
%\chapter*{Agradecimentos}
%Agradecimentos ...

% ------------------------------------------
% Resumen
%\chapter*{Resumen}
%Resumen en espanol ...

%\noindent \textbf{Palabras-clave:} palabra-clave1, palabra-clave2, palabra-clave3.

% ------------------------------------------
% Abstract
%\chapter*{Abstract}
%Abstract ...

%\noindent \textbf{Keywords:} keyword1, keyword2, keyword3.

% ------------------------------------------
% Indice
\tableofcontents    % imprime el indice
% ------------------------------------------
% Listas: abreviaturas, simbolos, figuras y cuadros

\listoffigures               % lista de Figuras
\listoftables                % lista de cuadros

% ---------------------------------------------------------------------------- %
% Capitulos
\mainmatter

%Definir el interlineado
%\singlespacing		% interlineado simple
\onehalfspacing		% interlineado 1.5
%\doublespacing		% interlineado doble 

%Lista de archivos .tex por capitulo
\input cap-01introduction
\input cap-02distribution
\input cap-03model
\input cap-04estimation
\input cap-05application
\input cap-06conclusion
\appendix
\chapter{Code used}
\label{ape:codigo}

% ---------------------------------------------------------------------------- %
% Bibliografia
\backmatter \singlespacing

\nocite{*} 
\bibliography{bibliografia}
\bibliographystyle{dcu}


\end{document}

% fin del archivo :-)