
\documentclass[11pt,oneside,a4paper]{book}

% ---------------------------------------------------------------------------- %

\usepackage[english]{babel}
\usepackage[latin1]{inputenc}
\usepackage[pdftex]{graphicx}           % para insertar figuras en formato pdf/png/jpg
\usepackage{color}
\usepackage{pifont}
\usepackage{amsfonts}
\usepackage{amssymb} 
\usepackage{amsmath}
\usepackage{setspace}				% espaciamento flexible
\usepackage[small,compact]{titlesec}	% encabezamiento de los titulos: menores y compactos
\usepackage{indentfirst}				% indentacion del primer parrafo
\usepackage[round]{natbib}			% libreria para la bibliografia
\usepackage{subfigure}				% uso de varias figuras en una sola
\usepackage[nottoc]{tocbibind}		% para que la bibliografia aparezca en el indice
\usepackage{setspace}
\usepackage{longtable}
\usepackage{lscape}
\usepackage{caption}
\usepackage[colorlinks=true,urlcolor=red,citecolor=green,linkcolor=blue]{hyperref}
\usepackage[a4paper,top=2.54cm, bottom=2.54cm, left=3cm, right=2.54cm]{geometry} %margenes

% ---------------------------------------------------------------------------- %
% Algunos comandos
\graphicspath{{./img/}}	% direccion donde se encontraran las figuras
\makeindex			% para crear el indice
\raggedbottom			% para no tener espacios extras en el texto
\listfiles			% genera una lista de los archivos utilizados durante la compilacion
\normalsize

% Para tener un tamaoo de fuente menor (para figuras e cuadros)
\newcommand{\captionfonts}{\small}
\makeatletter  % Allow the use of @ in command names
\long\def\@makecaption#1#2{%
  \vskip\abovecaptionskip
  \sbox\@tempboxa{{\captionfonts #1: #2}}%
  \ifdim \wd\@tempboxa >\hsize
    {\captionfonts #1: #2\par}
  \else
    \hbox to\hsize{\hfil\box\@tempboxa\hfil}%
  \fi
  \vskip\belowcaptionskip}
\makeatother   % Cancel the effect of \makeatletter

% para mejorar la presentacioon de las figuras en el texto
\renewcommand{\topfraction}{0.85}
\renewcommand{\textfraction}{0.1}
\renewcommand{\floatpagefraction}{0.75}

% comodidades
\newcommand{\e}[1]{{\mathbb E}\left[ #1 \right]}
\newcommand{\var}[1]{{\mathbb V}\left[ #1 \right]}

% ---------------------------------------------------------------------------- %
% Cuerpo del texto
\begin{document}
\frontmatter \onehalfspacing  % tipo de interlineado

% ---------------------------------------------------------------------------- %
% Carotula
\thispagestyle{empty}
\begin{center}
\vspace*{1cm}
\textbf{\Large{PONTIFICIA UNIVERSIDAD CAT\'OLICA DEL PER\'U }}\\
\vspace*{1.2cm}

\textbf{\Large{ESCUELA DE GRADUADOS}}\\
\vspace*{0.5cm}
\begin{center}
\includegraphics[scale=.25]{logoPUCP}
\end{center}
\vspace{0.5cm}

\textbf{\Large{``Endpoint-inflated beta-binomial regression\\for correlated count data''}}\\
\vspace{1.2cm}
\textbf{\large{TESIS PARA OPTAR POR EL GRADO DE MAGISTER EN\\
  ESTAD\'ISTICA}}\\
  
\vspace*{1.2cm}
\textbf{\large{Presentado por:}}\\
\vspace*{0.3cm}
\textbf{\large{Boris Manuel Fazio Luna}}\\
\vspace*{1.2cm}
\textbf{\large{Asesor: Victor Giancarlo Sal Y Rosas Celi}}\\

\vspace*{1.2cm}

\textbf{\large{
	Miembros del jurado:\\
	Dr. Nombre completo jurado 1 \\
	Dr. Nombre completo jurado 2 \\
	Dr. Nombre completo jurado 3  
	}} 
	   
\vspace*{1.2cm}
    
\normalsize{Lima, Julio 2018}
\end{center}

%-------------------------------------------
% Dedicatoria
%\chapter*{Dedicatoria}
%Dedicatoria...

%-------------------------------------------
% Agradecimentos
%\chapter*{Agradecimentos}
%Agradecimentos ...

% ------------------------------------------
% Resumen
%\chapter*{Resumen}
%Resumen en espanol ...

%\noindent \textbf{Palabras-clave:} palabra-clave1, palabra-clave2, palabra-clave3.

% ------------------------------------------
% Abstract
%\chapter*{Abstract}
%Abstract ...

%\noindent \textbf{Keywords:} keyword1, keyword2, keyword3.

% ------------------------------------------
% Indice
\tableofcontents    % imprime el indice
% ------------------------------------------
% Listas: abreviaturas, simbolos, figuras y cuadros

\listoffigures               % lista de Figuras
\listoftables                % lista de cuadros

% ---------------------------------------------------------------------------- %
% Capitulos
\mainmatter

%Definir el interlineado
%\singlespacing		% interlineado simple
\onehalfspacing		% interlineado 1.5
%\doublespacing		% interlineado doble 

%Lista de archivos .tex por capitulo
%% ------------------------------------------------------------------------- %%
\chapter{Introduction}
\label{cap:introduction}

%% ------------------------------------------------------------------------- %%
\section{Preliminary Considerations}
\label{sec:introduction_preliminary-considerations}

IDEAS

- 

Models that introduce inflation at any given value or values are a subtype of mixture models whereby a random distribution is mixed in with degenerate random variables on the same support. This type of models are useful in situations where two underlying processes can result in the same value being recorded, e.g. when asked for number of cigarrettes smokd in the past week, zero is a valid response from both non-smokers and smokers who did not do so recently. An important feature of inflation models is that separate regressions can be used to model the mixture proportions as well as the underlying random variable's behavior, which may provide insights into the different factors driving the behavior of the system under study.

The earliest model of this type was the zero-inflated Poisson (ZIP), put forth in \cite{lambert1992zero}, which was used in understanding the processes whereby manufacturing defects occur. In that setting, a zero may come from equipment in working order as well as defective equipment which may occassionally still output defect-free products. More recent applications of the ZIP model can be seen in \cite{wang2017availability}, where it is evaluated as one possible model for representing the number of times in which fruits and vegetables where recently consumed.

Depending on the framing of the question, however, models related to the Poisson distribution may not be appropriate. \cite{yen2011fruit} uses Poisson and negative binomial models to analyze the number of days in which fruit and vegetable consumption occured, out of the even in a week. This incorrectly places support over all non-negative integers, whereas the question only admits responses in the zero to seven range.

A model that can represent bounded counts with inflation at either endpoint, the endpoint-inflated binomial (EIB) was put forth by \cite{Tian2015}. Applications include simultaneous estimation of total, partial and null susceptibility to drugs or infectious agents, as may be obtained by recording number of organisms affected by a treatment or family members afflicted by disease. As referenced in the previous paragraph, the model is also applicable to survey questions where a bounded count response is requested.

Our present work provides an additional extension the EIB model by considering a beta-binomial distribution instead, which provides a versatile way to model overdispersion that is distinguishable and co-occuring with endpoint inflation.

%% ------------------------------------------------------------------------- %%
\section{Objectives}
\label{sec:introduction_objectives}

The overall objective of this thesis is to describe an endpoint-inflated beta-binomial (EIBB) regression model and show its application in a real world problem under a Bayesian paradigm.

Specifically, our goals are as follows:

\begin{itemize}
\item Provide a brief literature review of available models for analyzing inflated count data.
\item Describe the EIBB distribution and its parametrization as a regression model.
\item Implement a flexible framework for our model using $\text{R}$ and $\text{Stan}$ programming languages.
\item Conduct simulation studies to understand model behavior over the space of plausible parameters and at its boundaries.
\item Show its use in analyzing a real world dataset, including model diagnostics.
\end{itemize}
%% ------------------------------------------------------------------------- %%
\chapter{The endpoint-inflated beta-binomial distribution}
\label{cap:distribution}

In this chapter we show the steps to construct the endpoint-inflated beta-binomial distribution, examine its properties and define the parametrization that will be used through the rest of this paper.

We begin by introducing the simpler and familiar distributions that will be used as our building blocks.

%% ------------------------------------------------------------------------- %%
\section{Binomial distribution}
\label{sec:bino-dist}

When $X$ indicates the total number of successes in a series of $n \in \mathbb{N}^+$ independent dichotomous trials, each with probability $p \in [0,1]$, we say that it has a binomial distribution. The point mass function of a binomial random variable is given by

\begin{equation}
\label{binomial-pmf}
\begin{split}
f_{X}(x \mid p; n)
&= \binom{n}{x}p^x(1-p)^{n-x}\\
\end{split}
\end{equation}

with central moments

\begin{equation}
\label{binomial-moments}
\begin{split}
\e{X} = np, \quad \var{X} = np(1-p).
\end{split}
\end{equation}

The binomial upper bound $n$ will be assumed to be known throughout this paper and excluded from discussions of the distribution's parameters, though the opposite can hold in more general treatments.

%% ------------------------------------------------------------------------- %%
\section{Beta distribution}
\label{sec:beta-dist}

A random variable $Y \in (0,1)$ follows a beta distribution with parameters $\alpha, \beta > 0$ if its probability density function is given by

\begin{equation}
\begin{split}
f_{Y}(y \mid \alpha, \beta)
&= \frac{\Gamma(\alpha+\beta)}{\Gamma(\alpha)\Gamma(\beta)}y^{\alpha-1}(1-y)^{\beta-1},
\end{split}
\end{equation}

which central moments $\e{Y} = \alpha / (\alpha+ \beta)$ and $\var{Y} = \alpha \beta/[(\alpha+\beta)^2(\alpha+\beta+1)]$.

For brevity, expressions that take the form of the normalizing factor in the above pdf will, from this point on, be denoted through the beta function:

\begin{equation}
\label{beta-function}
\begin{split}
B(\alpha, \beta) = \frac{\Gamma(\alpha)\Gamma(\beta)}{\Gamma(\alpha+\beta)}
\end{split}
\end{equation}

In the regression context that will be developed later, it is more convenient to parametrize the beta distribution in terms of its mean $\mu \in (0,1)$ and a precision parameter $\phi > 0$. The resulting equivalence is  $\alpha = \mu\phi$ and $\beta = (1-\mu)\phi$ and the central moments are reexpressed as

\begin{equation}
\begin{split}
\e{Y} = \mu, \quad \var{Y} = \frac{\mu(1-\mu)}{\phi+1}.
\end{split}
\end{equation}

%% ------------------------------------------------------------------------- %%
\section{Beta-binomial distribution}
\label{sec:bbin-dist}

Beta random variables have support over $(0,1)$, which includes all values that are allowed for the $p$ parameter in a non-degenerate binomial random variable. If observations on a binomial random variable $X$ are thought to come from a randomly drawn, beta-distributed $p=P$, then the marginal distribution of $X$ is beta-binomial. Notationally, the relationship is

\begin{equation}
\begin{split}
X \mid P &\sim \text{Binomial}(P; n)\\
P &\sim \text{Beta}(\mu, \phi)\\
\Rightarrow X &\sim \text{Beta-binomial}(\mu, \phi; n).
\end{split}
\end{equation}

With $B(\alpha,\beta)$ as defined in \ref{beta-function}, the beta-binomial pmf is

\begin{equation}
\begin{split}
f_X(x \mid \mu, \phi; n) = \binom{n}{x}\frac{B(x+\mu\phi, n - x + (1-\mu)\phi)}{B(\mu\phi, (1-\mu)\phi)},
\end{split}
\label{betabinomial-pmf}
\end{equation}

with central moments

\begin{equation}
\begin{split}
\e{X} = n\mu, \quad \var{X} = n\mu(1-\mu)\frac{\phi+n}{\phi+1}.
\end{split}
\end{equation}

Comparing the above expressions with those for the binomial \ref{binomial-moments}, it can be seen that the meas take the same form, both being governed by a single centrality parameter on the unit iterval. For identical values of $\mu$ and $p$, it can be seen that the beta-binomial variance will exceed that of the binomial by a factor of $(\phi+n)/(\phi+1)$. REFERENCIA A FIGURA CON BINOM VS BETA-BINOM.

%% ------------------------------------------------------------------------- %%
\section{Endpoint-inflated beta-binomial distribution (EIBB)}
\label{sec:eibb-dist}

A random variable $X$ that follows an EIBB distribution behaves as

\begin{equation}
\begin{split}
X \sim
\begin{cases}
\text{Degenerate}(0) \qquad &\text{ with probability } p^{_0},\\
\text{Beta-binomial}(\mu, \phi; n) \qquad &\text{ with probability } p^{_1},\\
\text{Degenerate}(n) \qquad &\text{ with probability } p^{_2}.
\end{cases}
\end{split}
\end{equation}

This is simply a mixture involving the beta-binomial distribution and two degenerate distributions at $0$ and $n$, the endpoints for the beta-binomial.

In order to write the pmf, we define $I_c(x)$ to be the indicator function for point $c$ and introduce $Y \sim \text{Beta-binomial}(\mu, \phi; n)$. Then the EIBB random variable $X$ has a distribution given by

\begin{equation}
\begin{split}
f_{X}(x \mid \mu, \phi, p^{_0}, p^{_1}, p^{_2}; n)
&= p^{_0}I_0(x) + p^{_1} f_{Y}(x) + p^{_2}I_n(x)\\
&=	\begin{cases}
p^{_0} + p^{_1} {B(\mu\phi, n+(1-\mu)\phi) \over B(\mu\phi,(1-\mu)\phi)}, & \mathrm{if}\; x=0\\
p^{_1} {n \choose x} {B(x+\mu\phi, n-x+(1-\mu)\phi) \over B(\mu\phi,(1-\mu)\phi)}, & \mathrm{if}\; x=1,...,n-1\\
p^{_2} + p^{_1} {B(n+\mu\phi, (1-\mu)\phi) \over B(\mu\phi,(1-\mu)\phi)}, & \mathrm{if}\; x=n\\
0, & \mathrm{otherwise}
	\end{cases}
\end{split}
\label{densidad}
\end{equation}
	
\noindent where the shape parameters $\alpha > 0$ and $\beta > 0$ are unknown, as are $p^{_0} \in (0, 1)$ and $p^{_1} \in (0, 1)$, which represent the proportions of left an right endpoints that occur in excess of that which would be expected if the generating distribution was beta-binomial alone. Clearly, $p^{_2} = 1 -  (p^{_0}+p^{_1})$.  The right endpoint, $n$, is assumed to be known.\\

\begin{figure}
  \includegraphics[width=\linewidth]{eibb2.png}
  \caption{The EIBB distribution for three combinations of mean and precision parameters with fixed mixture proportions $p^{_0} = p^{_1} = 0.25$}
  \label{fig:eibb}
\end{figure}
%% ------------------------------------------------------------------------- %%
\chapter{The endpoint-inflated beta-binomial regression model}
\label{cap:model}

In this chapter we introduce our formulation of the EIBB regression model. We will first present the full model likelihood and a set of suggested priors for Bayesian inference. We finish with a brief discussion of applicable model comparison criteria.

%% ------------------------------------------------------------------------- %%
\section{Model definition}
\label{sec:defmodel}

Let $Y_i = \begin{pmatrix}Y_{i1} & ... & Y_{iN_i}\end{pmatrix}^\top$, $i = 1,...,N$ be a set of independent response vectors where

\begin{equation}
Y_{ij} \sim \mathrm{EIBB}(\mu_{ij}, \phi_{ij}, p^{_0}_{ij}, p^{_1}_{ij}, p^{_2}_{ij}; n_{ij}).
\end{equation}
	
The dependence of model parameters on covariates is given by the following general structure:

\begin{equation}
\begin{split}
\mu_{ij} &= h_1(x_{ij}^\top \beta + z_{ij}^\top b_i)\\
\phi_{ij} &= h_2(\hat{x}_{ij}^{\top} \gamma + \hat{z}_{ij}^{\top} g_i)\\
\begin{pmatrix}p_{ij}^{_0} & p_{ij}^{_1} & p_{ij}^{_2}\end{pmatrix}^\top &= h_3(\dot{x}_{ij}^{\top} \dot{\delta}+\dot{z}_{ij}^{\top} \dot{d}_i, \ddot{x}_{ij}^{\top} \ddot{\delta}+\ddot{z}_{ij}^{\top} \ddot{d}_i)
\end{split}
\label{regression-model}
\end{equation}

\noindent where $\beta, \gamma, \dot{\delta}, \ddot{\delta}$ are vectors of fixed effects and $b_i, g_i, \dot{d}_i, \ddot{d}_i$ are vectors of random effects. The vectors $x_{ij}, \hat x_{ij}, \dot x_{ij}, \ddot x_{ij}$ hold covariate values.

In order to remain within parameter space, the link functions must be defined so that

\begin{equation}
\begin{split}
h_1:&\: \mathbb R\; \to (0,1)\\
h_2:&\: \mathbb R\; \to \mathbb R^+\\
h_3:&\: \mathbb R^2 \to (0,1)^3\\
&\: \forall j,k \in \mathbb R \left( \vec 1 \cdot h_3(j,k) = 1 \right).
\end{split}
\label{link-definitions}
\end{equation}

We will take the usual selecton of logit for $h_1$ and log for $h_2$. To motivate our choice for the link function on the mixture proportions, $h_3$, note that the mixture components can be ordered in terms of their expected values: $0<n\mu<n$, for the first, second and third components respectively. We will relax this interpretation soon, but for now, see that the latent $Z$ in \ref{eibb-distribution} which selects the mixture components can be regarded as an ordered categorical variable. Then the ordered probit model gives us a starting point by which to introduce a linear predictor on its parameters $p_{ij}^{_0}$, $p_{ij}^{_1}$, $p_{ij}^{_2}$.

If $\Phi(.;\mu',\sigma'^2)$ is the cumulative distribution function of a normal with mean $\mu'$ and variance $\sigma'^2$ and $c_0$, $c_1$ are two real numbers such that $c_0 < c_1$, the ordered probit regression model is given by

\begin{equation}
\begin{split}
p^{_0} &= \Phi(c_0;\mu'_{ij},\sigma'^{2}) \\
p^{_1} &= \Phi(c_1;\mu'_{ij},\sigma'^{2}) - \Phi(c_0;\mu'_{ij},\sigma'^{2})\\
p^{_2} &= 1 - \Phi(c_1;\mu'_{ij},\sigma'^{2})\\
\mu'_{ij} &= \dot{x}_{ij}^{\top} \dot{\delta}+\dot{z}_{ij}^{\top} \dot{d}_i
\end{split}
\label{eq:oprobit}
\end{equation}

where $x'^\top \nu$ holds the fixed effects terms and $z'^\top v$ are any random effects terms. Given the restriction $\sum_s p_{ij}^{_s} = 1$, we see that the above system relates two free parameters with four unknowns. Therefore some values must be fixed to obtain a unique solution. We choose to set $c_0 = 0$ and $c_1 = 1$. Figure \ref{fig:oprobit} illustrates this setup.

The assumption of a fixed $\sigma'^{2}$ can now be relaxed by setting

\begin{equation}
\begin{split}
\sigma'^{2}_{ij} &= log(\ddot{x}_{ij}^{\top} \ddot{\delta}+\ddot{z}_{ij}^{\top} \ddot{d}_i).
\end{split}
\label{eq:oprobit2}
\end{equation}

This retains the flexibility of the more common softmax link function while offering a latent variable interpretation for the inflation behavior.

\begin{figure}
  \includegraphics[width=\linewidth]{oprobit.png}
  \caption{The black curve shows a $\mu'$, $\sigma'^{2}$ normal cdf and how the cutpoints $c_0$, $c_1$ are used to retrieve the vector of probabilities. The corresponding normal pdf is shown in gray.}
  \label{fig:oprobit}
\end{figure}

%% ------------------------------------------------------------------------- %%
\section{Bayesian Inference}

As recommended in \cite{Gelman2013}, we use weakly informative priors for all coefficients in the model. For all data where it would seem reasonable to apply this model, the linear predictor for unit interval parameters, under the link functions discussed above, should contain coefficients with magnitudes in the single digits (provided covariates are first standardized). Therefore, zero-centered normal priors with $\sigma=10$ are a reasonable option.

Priors associated to dispersion/precision parameters should keep mass away from zero and allow large values. Cauchy priors fulfill this requirement for coefficients in the linear predictors, but an improper prior for the intercepts is best, as there is no reason to assume they are located around zero.

Though sometimes a full set of improper priors is used as a way of "letting the data speak for itself",  note that such a choice does not guarantee a proper posterior will result. \cite{Tak2015} Briefly, one should verify that interiors group exist, i.e. all the data should not be located at the endpoints, and that there are at least as many observations as there are parameters. If proper priors are employed, these issues are avoided altogether.

%% ------------------------------------------------------------------------- %%
\section{Measures for model assessment}
\label{sec:modcomp}

\subsection{Information criteria}

Information criteria provide a way to compare the relative predictive prowess of two or more models. Suitable choices under a Bayesian paradigm include Deviance Information Criterion (DIC), Widely Applicable Information Criterion (WAIC), Expected Akaike Information Criterion (EAIC) and Expected Bayesian Information Criteria (EBIC). The key quantity used to calculate them is the deviance, defined as

\begin{equation}
\mathcal{D}(L)=-2\log\{L\}
\end{equation}

\noindent where $L$ is the model's likelihood.

We now provide the formulas for the information criteria, using a bar to denote posterior means.

\begin{itemize}
\item $\text{DIC}=\mathcal{D}(\bar L) + 2(\bar{\mathcal{D}} (L) - \mathcal{D} (\bar L))$
\item $\text{WAIC}=\mathcal{D}(\bar L) + \overline{\text{Var}}(\log L)$
\item $\text{EAIC}=\mathcal{D}(\bar L) + p$
\item $\text{EBIC}=\mathcal{D}(\bar L) + p \times \log (n)$
\end{itemize}

In the last two equations, note that $p$ and $n$ stand for number of parameters and observations, respectively.

Because deviance is minimized when parameters allow a perfect fit to the data, models that score lower are to be preferred. The second term in the sum of each criterion are corrections which penalize models for complexity, such that from two models with equal deviance, the one which uses less parameters is favored.

Some issues are the size of what may be considered meaningful differences and that, as these are relative measures of performance, selecting the model with lowest information criteria from a set of candidates provides no guarantee that the model is in fact a good fit for the data at hand. The next section provides a test that may be used to evaluate overall fit, independently of alternative reference models.

\subsection{Bayesian $\chi^2$}

\cite{johnson2004bayesian} proposes a test for a goodness-of-fit testing in a Bayesian setting based on the classical $\chi^2$ test. The test statistic for a discrete random variable model is given by

\begin{equation}
R^B(\widetilde{\theta}) = \sum_{k=1}^K \left[{m_k - Np_k(\widetilde{\theta}) \over \sqrt{Np_k(\widetilde{\theta})}}\right]^2
\end{equation}

where $p_k(\widetilde{\theta}) = \frac{1}{N} \sum_{j=1}^N\sum_{y\in \text{bin}k}f_j(y \mid \widetilde{\theta})$ denotes the sum of pmfs $f_j$ evaluated at each observation, conditioned on $\widetilde{\theta}$, a single draw from the posterior parameter vector. The $K$ bins over which the sum takes place correspond to the possible values the random variable may take and $m_k$ is the actual number of such values observed.

The distribution of $R^B(\widetilde{\theta})$ converges to a $\chi^2$ with $K-1$ degrees of freedom as $n \rightarrow \infty$, provided regularity conditions are met and parameter draws are independent. Although this last requirement is violated when repeated draws from the posterior are made based on the sample data sample, the approximation is often good enough. 

To measure fit, one simply draws samples of $R^B$ and a  $\chi^2_{K-1}$ random variable, then compares the proportion of test statistic values which exceed the reference distribution. Large deviations from a value of $0.5$ are then indicative of bad model fit, without need for providing another reference model.

The use of both types of measures will be illustrated in the simulation and application sections which follow.
%% ------------------------------------------------------------------------- %%
\chapter{Simulation study}
\label{cap:simulation}

%% ------------------------------------------------------------------------- %%
\section{Parameter grid}
\label{sec:prior}

%% ------------------------------------------------------------------------- %%
\section{Results}
\label{sec:simu}
%% ------------------------------------------------------------------------- %%
\chapter{Applications with real data}
\label{cap:applications}

In this chapter, we show the inferences obtained by applying the EIBB regression model to a real dataset and compare its performance to simpler alternatives.

%% ------------------------------------------------------------------------- %%
\section{Data}
\label{sec:applications-data}

The \textit{Encuesta Demogr\'afica y de Salud Familiar} (ENDES) is a yearly national survey conducted by the Peruvian government to provide information on demographic dynamics and health status of the population. Households are sampled in a stratified two-stage scheme and information on household members is gathered, with some questionnaires being restricted based on the person's age and sex.

We will use data from the 2017 round, which included the \textit{Cuestionario de Salud}, a health-focused questionnaire that includes questions on behaviors linked to chronic disease and is applied to people from both sexes and ages 15 onwards. Specifically, we take as our outcome variable the response to item 219, which asks the respondent how many days out of the last seven they ate vegetable salad. By design, the response is discrete and bounded between 0 and 7. Additionally, the empirical distribution of the responses suggests that inflation is present in at both endpoints. Thus, an EIBB regression appears to be a suitable model choice.

%% ------------------------------------------------------------------------- %%
\section{Model structure}
\label{sec:applications-structure}

%% ------------------------------------------------------------------------- %%
\section{Results}
\label{sec:applications-results}



%% ------------------------------------------------------------------------- %%
\section{Role preference in MSM}
\label{sec:sexmsm}
%% ------------------------------------------------------------------------- %%
\chapter{Conclusions}
\label{cap:concl}
\appendix
\chapter{Code used}
\label{ape:codigo}

% ---------------------------------------------------------------------------- %
% Bibliografia
\backmatter \singlespacing

\nocite{*} 
\bibliography{bibliografia}
\bibliographystyle{dcu}


\end{document}

% fin del archivo :-)